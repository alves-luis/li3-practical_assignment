\documentclass[a4paper]{article}

\usepackage[utf8]{inputenc}
\usepackage[portuges]{babel}
\usepackage{graphicx}
\usepackage[colorlinks = true, urlcolor = blue, linkcolor = black]{hyperref}
\usepackage{tabto}
\usepackage[simplified]{pgf-umlcd}
\renewcommand{\umldrawcolor}{black}

\begin{document}

\title{Relatório do Trabalho Prático de LI3}
\author{Grupo 42\\
\\
João Queirós (A82422)\\
\\
José Costa (A82136)\\
\\
Luís Alves (A80165)\\
\\
Miguel Carvalho (A81909)}
\date{\today}

\maketitle
\tableofcontents

\section{Introdução}

	\subsection{Contextualização}
		\tab Este projeto foi desenvolvido no âmbito da Unidade Curricular
		de \textit{Laboratórios de Informática III}, no qual analisaremos e processaremos
		informação do \href{https://pt.stackoverflow.com}{\textit{\textbf{Stack Overflow}}},
		\href{https://android.stackexchange.com}{\textit{\textbf{Android}}} e
		\href{https://askubuntu.com}{\textit{\textbf{Ask Ubuntu}}}, recorrendo
		à linguagem de programação Java.


	\subsection{Objetivos}
		\tab Após realizarmos o trabalho em C, propusemo-nos a melhorar a modularização
		deste, tal como reduzir ao mínimo as interdependências entre classes.
		\par Ainda assim, tivemos sempre em mente a obtenção de reduzidos tempos de resposta
		para cada \textit{query}.

\section{Estrutura de Dados}
	\tab De forma a organizarmos as nossas classes pelos packages previamente criados,
	optamos por colocar os tipos básicos de dados no package \texttt{li3}, assim como
	as interfaces que definiam métodos a implementar pela classe que armazenasse os
	dados da aplicação (\texttt{Community}).
	\par Criamos um package extra \texttt{exceptions} onde colocamos, naturalmente,
	algumas exceptions que criamos para o projeto.
	\par No package \texttt{common}, colocamos todos os \textit{Comparators} usados
	nas diversas classes, enquanto que no package \texttt{engine} colocamos a classe
	de parsing (\texttt{Parser}), as classes respetivas a cada \textit{query}, bem como a
	classe que implementa a \texttt{TADCommunity} (\texttt{ForumsModel}) e a classe
	que funciona como base de dados da aplicação (\texttt{Community}).

	\subsection{Tipos Básicos}

		\begin{figure}
			\centering

		\begin{tikzpicture}
			\begin{interface}[text width = 3cm]{Identifiable}{8,-3}
				\attribute{public long getId()}
			\end{interface}

			\begin{class}[text width = 2cm]{Post}{2,-3}
				\implement{Identifiable}
			\end{class}

			\begin{class}[text width = 2cm]{Question}{0,0}
				\inherit{Post}
			\end{class}

			\begin{class}[text width = 2cm]{Answer}{4,0}
				\inherit{Post}
			\end{class}

			\begin{class}[text width = 2cm]{Tag}{8,0}
				\implement{Identifiable}
			\end{class}

			\begin{class}[text width = 2cm]{User}{12,0}
				\implement{Identifiable}
			\end{class}


		\end{tikzpicture}
		\caption{Diagrama dos Tipos Básicos} \label{diag:basic}
		\end{figure}

		\subsubsection{Post}
			\tab Um \texttt{Post} é um objeto simples que contém informações comuns a \texttt{Question}
			e a \texttt{Answer}: título, id, data de criação e id do criador. Este objeto
			é a superclasse das classes acima referidas.

		\subsubsection{Question}
			\tab Uma \texttt{Question} é subclasse de \texttt{Post}, e contém uma lista
			com os nomes das tags dessa questão bem como o atributo \textit{AnswerCount}.

		\subsubsection{Answer}
			\tab Uma \texttt{Answer} é subclasse de \texttt{Post}, e contém várias informações
			adicionais relativamente a este: id da questão que responde, o seu score e o nº
			de comentários.

		\subsubsection{Tag}
			\tab Uma \texttt{Tag} é um objeto que define os tipos de tags que uma questão
			pode ter, por isso contém: o nome da tag e o id associado.

		\subsubsection{User}
			\tab Um \texttt{User} é um objeto que representa um utilizador no sistema, e contém
			as seguintes informações: uma pequena bio, nome, id, reputação e nº de posts.

	\subsection{Armazenamento de Dados}\label{sec:community}
		\tab De forma a posteriormente podermos consultar os dados de forma eficiente,
		optou-se por não criar clones dos objetos indiscriminadamente, já que estes,
		após serem carregados, não seriam mais alterados, já que todas as queries são
		meramente de consulta. Isto permitiu uma poupança significativa na memória ocupada,
		bem como no trabalho extra que seria necessário por parte do CPU para clonar todos
		os objetos. Ainda assim, consideramos que os dados ficam devidamente encapsulados,
		já que apenas permitimos o acesso via métodos públicos com um comportamento bem
		definido (as variáveis de instância permanecem privadas).
		\par Assim sendo, criamos uma classe \texttt{Community} que agrega os dados
		da seguinte forma:
			\begin{itemize}
				\item Map de id para \texttt{User}
				\item Map de id para \texttt{Post}
				\item Map de id de uma \texttt{Question} para um Set de \texttt{Answer} (ordenado por ordem
				cronológica inversa)
				\item Map de id de um \texttt{User} para um Set de \texttt{Post} (ordenado por ordem
				cronológica inversa)
				\item Lista de \texttt{User} ordenado por ordem decrescente de nº de \texttt{Post}
				\item TreeMap de \texttt{LocalDateTime} para um Set de \texttt{Post}, ordenado por ordem
				cronológica inversa
				\item Map de nome de uma \texttt{Tag} para a respetiva \texttt{Tag}
			\end{itemize}
		\par Os Maps foram usados de forma a obter em tempo constante um determinado objeto/conjunto
		de objetos de acordo com o seu identificador. Quanto à lista, esta foi usada pois, aquando
		do momento de \textit{Loading}, colocamos primeiro os utilizadores na \texttt{Community}. No
		entanto, não sabemos \textit{à priori}, o seu número de posts. Por isso, esse número é apenas
		incrementado quando procedemos ao carregamento dos posts. Findo esse carregamento, podemos
		usar um \textit{Comparator} na Lista, de forma a ordenarmos por número de posts.
		\par Quanto ao TreeMap, este foi usado por disponibilizar o método \texttt{subMap} que
		se revelou muito eficiente na obtenção de posts entre determinados intervalos de tempo.


\section{Carregamento de Dados}
	\subsection{Parsing}
		\tab Para efetuarmos o parsing dos ficheiros .xml, optamos por usar a
		\textit{Streaming API for XML} (StAX), já que permitia não carregar o ficheiro
		.xml todo para memória, mas sim carregar "row a row", uma melhoria relativamente
		ao parsing feito em C.
		\par Para isso, criamos uma classe \texttt{Parser} que apenas possui métodos
		de classe, para retirar informação útil dos ficheiros \textit{Posts.xml}, \textit{Users.xml}
		e \textit{Tags.xml}, devolvendo listas com o conteúdo de acordo com os respetivos
		objetos (\texttt{Post}, \texttt{User} e \texttt{Tag}).

	\subsection{Loading}
		\tab Ao efetuarmos o \textit{loading} dos ficheiros, limitamo-nos a pegar nas listas resultantes
		dos métodos de classe da classe \texttt{Parser}, e a colocar cada um desses objetos
		no objeto \texttt{Community}. Ora esse objeto possui um conjunto de métodos (privados)
		que relacionam os posts, users e tags de acordo com as variáveis de instância definidas
		(ver secção \ref{sec:community}).
		\par O \textit{loading} dos ficheiros é feito pela seguinte ordem: primeiro são
		armazenados os users, depois as tags e por fim os posts.

\section{Modularização}
	\tab De forma a ser possível desenvolver o código concorrentemente pelos 4 elementos
	do grupo, optamos por partir o problema o máximo possível para não haver interdependências ou,
	havendo-as, serem minimizadas, nomeadamente através da definição de interfaces que sinalizavam
	métodos que posteriormente deveriam ser implementados para completar a funcionalidade de uma
	determinada \textit{query}. Assim sendo, cada \textit{query} também tem a sua própria classe com
	métodos de classe que trabalham sobre os dados disponibilizados pela \texttt{Community}.
	\par Tendo isto em consideração, criamos os seguintes módulos fundamentais, que poderiam
	ser desenvolvidos simultaneamente:
		\begin{itemize}
			\item Parsing dos ficheiros XML
			\item Tipos básicos
			\item Definição da estrutura que albergasse os dados e métodos de disponibilização
			desses mesmos dados
			\item Queries
		\end{itemize}

\section{Queries}
	\subsection{InfoFromPost (1)}
		\tab Utilizando o método \texttt{getPost()} que recebe como argumento o id do \texttt{Post}
		que queremos consultar no objeto \texttt{Community} que contém os dados (daqui em diante
		todos os métodos chamados a um objeto não seja explicitado, assumem-se que sejam chamados
		ao \texttt{Community}).
		\par Caso seja devolvido um \texttt{Post}, verifica-se se é uma \texttt{Answer} ou \texttt{Question},
		pois se for a primeira, o título e nome devolvidos são os da questão e criador respetivos; se for a
		segunda, é devolvido o título e nome da resposta e do criador da resposta. Caso o \texttt{Post} não exista,
		ou não possua título/nome, é passado null como resultado ao \texttt{Pair}.

	\subsection{TopMostActive (2)}
	 \tab Utilizando o método \texttt{getUsersByNumberOfPosts()} que recebe como argumento
	 o número de utilizadores, os \texttt{User} são transformados no seu respetivo id e é
	 devolvida a lista com esses ids. O método primeiramente referenciado, limita-se a
	 passar para uma nova lista de tamanho máximo N a lista ordenada de utilizadores
	 de acordo com o número de posts. Refere-se novamente que, por não se alterar
	 o objeto, se passa a referência ao objeto que está guardado na \texttt{Community}.

	\subsection{TotalPosts (3)}
		\tab Nesta \textit{query}, inicialmente havíamos pensado em implementar uma estratégia
		que minimizasse o uso de computações específicas à \textit{query} dentro da classe \textit{Community}.
		Assim sendo, optamos por obter a lista de posts dentro de um intervalo e, posteriormente,
		iterá-la e contar se eram instâncias de \texttt{Question}, ou de \texttt{Answer}. No entanto, isto implicava
		que percorrêssemos essa lista de posts mais que uma vez. Optamos por uma outra abordagem, que se enuncia de seguida.
		\par Sendo o tempo de execução obtido demasiado elevado, optamos por, ao coletar os posts dentro
		de um intervalo na classe \texttt{Community}, contar de imediato se eram uma \texttt{Question}, ou uma \texttt{Answer}. Verificou-se um aumento
		de performance na ordem das 4 vezes (análise mais detalhada na secção \ref{ch:otimizacao}).

	\subsection{QuestionsWithTag (4)}
		\tab Inicialmente, utilizamos o método \texttt{filterQuestionByInterval()} que recebe
		como argumentos as datas de início e de fim da \textit{query} de forma a obtermos
		uma lista com \texttt{Question}s feitas nesse intervalo. De seguida, usando uma \texttt{stream},
		filtramos as questões de forma a de seguida podermos ordenar as que contêm a tag dada
		por ordem cronológica inversa.

	\subsection{GetUserInfo (5)}
		\tab Sendo esta \textit{query} bastante simples, a nossa estratégia passou por
		obter a bio de um utilizador através do método \texttt{getBio()}, que dado um id,
		verifica se o utilizador existe na \texttt{Community} e, se existir devolve a sua bio.
		Caso não exista, ou não possua bio, devolve \textit{null}.
		\par Quanto aos 10 últimos posts, recorrendo ao método \texttt{get10LatestPosts()} que devolve
		os 10 posts mais recentes de um utlizador dado o seu id, obtemos uma lista com os respetivos posts, já que
		armazenamos para cada utilizador, um Set já ordenado por ordem cronológica inversa dos seus
		posts.

	\subsection{MostVotedAnswers (6)}
		\tab Utilizando o método \texttt{filterAnswerByInterval()}, que recebe um intervalo
		de tempo e devolve todas as \texttt{Answers} feitas nesse período, apenas necessitamos
		de, recorrendo a uma \texttt{stream}, utilizar o método \texttt{sorted()} de acordo
		com um comparador de \texttt{Answers} por \textit{score} e limitar o tamanho
		da stream a N. É por fim coletada para uma lista de ids das \texttt{Answers}.

	\subsection{MostAnsweredQuestions (7)}
		\tab Inicialmente utilizamos o método \texttt{filterQuestionByInterval()} que, tal como
		os métodos com nome similar usados anteriormente, devolve uma lista com as questões
		realizadas no intervalo de tempo dado como argumento.
		De seguida, iteramos todas essas questões e colocamos num HashMap o id de cada questão
		como chave, e o número de respostas como valor. De seguida, fazemos \texttt{stream}
		do entrySet e usando o comparator \texttt{ComparatorLongIntEntryReverseInt} (que, dadas
		duas entries, devolve a com maior valor em primeiro lugar, e se os valores forem iguais, coloca
		a que tem maior key em primeiro lugar) ordena-se a stream limitando-a a N e coletando
		os ids das questões para uma lista.

	\subsection{ContainsWord (8)}
		\tab Para esta \textit{query}, utilizamos uma versão do método \texttt{filterQuestionByInterval()}
		que não recebe nenhum argumento e, por isso, retorna todas as questões presentes na
		\texttt{Community}. De seguida, utilizando uma \texttt{stream}, filtram-se as questões
		que contêm um título que não seja nulo e que contenha a palavra passada como argumento
		da \textit{query}, recorrendo ao método \texttt{contains()} da classe \texttt{String}.
		\par De seguida, é feita uma ordenação por ordem cronológica inversa limitada a N
		e são coletados os ids das questões para uma lista.


	\subsection{BothParticipated (9)}
		\tab Para solucionarmos esta \textit{query}, criamos dois sets: um para as questões
		em que participou o \texttt{User} 1, e outro para o \texttt{User} 2. As questões
		nas quais cada \texttt{User} participou são dadas pelo método \texttt{getQuestions()},
		que dado um id, devolve um set (que pode ser vazio) das questões nas quais um utilizador
		interveio, quer tenha sido via \texttt{Question}, ou \texttt{Answer}.
		\par De seguida, colocamos em \texttt{stream} o set de questões do \texttt{User} 1 e
		filtramos aqueles que também fazem parte do set de questões do \texttt{User} 2, através
		do método \texttt{contains()} da classe \texttt{Set}. De seguida, ordenam-se as questões
		por id e limita-se o tamanho a N, coletando os ids das questões numa lista.

	\subsection{BetterAnswer (10)}
		\tab Começamos a solucionar esta \textit{query} obtendo o set de respostas de uma
		dada pergunta através do método \texttt{getAnswers()}. A partir daí, para cada
		resposta é calculado o seu valor e adicionado a um TreeMap, que associa a cada
		\textit{score}, um set de ids de \texttt{Answer}, ordenado por ordem decrescente.
		\par Por fim, retira-se o valor à cabeça do TreeMap e é o id pretendido.

	\subsection{MostUsedBestRep (11)}
		\tab Sendo esta a \textit{query} mais complexa (e que mais dados necessita de cruzar), foi
		a que conseguimos um pior tempo de execução. O processo para obter o resultado é o que se
		enuncia a seguir.
		\par Inicialmente, é obtido o Set de \texttt{Questions} que foram publicadas no intervalo dado.
		A partir desse Set, foi criada uma lista de \texttt{Users} num método auxiliar (\texttt{usersThatParticipated()})
		que publicaram esses \texttt{Posts}, sendo depois essa Lista ordenada por reputação e truncada a N.
		\par Tendo já a lista final de N utilizadores com melhor reputação, por cada utilizador nessa lista,
		obtemos o Set de posts desse \texttt{User}, sendo que caso o \texttt{Post} seja uma
		\texttt{Question} e o primeiro Set de \texttt{Posts} obtido contiver este \texttt{Post},
		então é adicionado o id da \texttt{Tag} e o número de vezes que foi usada a um
		HashMap.
		\par Por fim, as \textit{entries} desse Map são colocadas em \texttt{stream} e
		ordenadas de acordo com os valores (e, caso os valores sejam iguais, por ordem
		crescente de ids), sendo coletados os ids para uma lista truncada a N.

\section{Análise de Performance}\label{ch:otimizacao}
	\tab Num computador portátil com 8GiB de RAM e um processador Intel® Core™ i7-3630QM @ 2.40GHz com
	8 cores,
	obtivemos os mesmos resultados que os de referência (atualização 3), tendo registado
	os tempos na tabela abaixo.

	\begin{table}[]
		\label{tab:times}
		\centering

		\begin{tabular}{c|c|c|}
		\cline{2-3}
		& \multicolumn{2}{c|}{Tempos} \\ \hline
		\multicolumn{1}{|c|}{Queries} & Parâmetro 1 & Parâmetro 2 \\ \hline
		\multicolumn{1}{|c|}{1} & 0ms & 1ms \\ \hline
		\multicolumn{1}{|c|}{2} & 7ms & 7ms \\ \hline
		\multicolumn{1}{|c|}{3} & 7ms & 25ms \\ \hline
		\multicolumn{1}{|c|}{4} & 28ms & 25ms \\ \hline
		\multicolumn{1}{|c|}{5} & 1ms & 1ms \\ \hline
		\multicolumn{1}{|c|}{6} & 14ms & 7ms \\ \hline
		\multicolumn{1}{|c|}{7} & 7ms & 68ms \\ \hline
		\multicolumn{1}{|c|}{8} & 148ms & 120ms \\ \hline
		\multicolumn{1}{|c|}{9} & 3ms & 2ms \\ \hline
		\multicolumn{1}{|c|}{10} & 1ms & 1ms \\ \hline
		\multicolumn{1}{|c|}{11} & 25ms & 145ms \\ \hline
		\end{tabular}
		\caption{Tempos registados}
	\end{table}

	\par Para além disso, notamos algumas diferenças quando implementamos diferentes
	abordagens nalgumas \textit{queries}.
	\par Na \textit{query} 3, em vez de pedirmos os posts entre datas por parte
	da nossa TCD, e depois iterá-los e verificar se eram \texttt{Question} ou \texttt{Answer},
	optamos por na própria \texttt{Community}, ao recolher os posts, devolver imediamente
	o par indicativo do número de perguntas e de respostas. Isto permitiu passar de 100ms,
	para 25, no parâmetro 2.
	\par Quanto às \textit{queries} que necessitavam de posts dentro de um dado intervalo,
	inicialmente havíamos optado por guardar esses posts num Set ordenado por data. No entanto,
	obter um subconjunto não se revelou muito eficiente, e por isso optamos por usar um
	TreeMap, que graças ao método \texttt{submap} permitia obter o range desejado
	mais rapidamente. Após implementarmos esta mudança na \texttt{Community}, verificamos
	que o tempo de execução de cada \textit{query} que requeria posts dentro de um intervalo
	diminui para cerca de 1/6 do tempo original com um Set.
	\par Uma nota também para o facto de a \textit{query} 8 ser a mais demorada, já que
	não organizamos o título de cada post de uma forma específica (dicionário, por exemplo)
	de forma a consultar se a palavra estava contida no título de forma ainda mais eficiente.

	\section{Conclusão}
		\tab Após a conclusão deste projeto, verificamos que este ficou com muito menos
		interdependências que o anterior em C. Para além disso, durante o desenvolvimento,
		notamos que este podia ser feito mais concorrentemente sem tantos problemas de
		\textit{merge}. Assim sendo, pensamos que cumprimos o planeado, tendo obtido
		bons tempos de execução em cada \textit{query}, boa modularização e código
		simples de modificar e fazer manutenção.

\end{document}
